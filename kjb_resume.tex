%%%%%%%%%%%%%%%%%%%%%%%%%%%%%%%%%%%%%%%%%%%%%%%%%%%%%%%%%%%%%%%%%%%
%% 
%% Kevin Berry's resume
%%   - based off work by Yisong Yue and Michael DeCorte
%%    (Template from http://www.yisongyue.com/resume/)
%%
%%%%%%%%%%%%%%%%%%%%%%%%%%%%%%%%%%%%%%%%%%%%%%%%%%%%%%%%%%%%%%%%%%%



%%
%% The following code sets up the document formatting
%%

%this assumes that res_yy.sty is in some path
\documentstyle[hyperref, margin, line]{res_yy}

\hypersetup{backref,pdfpagemode=Full,colorlinks=true,backref}

\addtolength{\oddsidemargin}{-0.45in}
\addtolength{\voffset}{-0.30in}
\addtolength{\textwidth}{1.00in} \addtolength{\textheight}{1.50in}

\renewcommand{\namefont}{\LARGE\emph}

%%
%% The following code defines some macros for terms which have raised font
%% (ie 4\fourth would result 4th with the 'th' raised (superscripted)
%%

\def\Cplusplus{{\rm C\raise.5ex\hbox{\small ++}}}
\def\CSharp{{\rm C\raise.5ex\hbox{\small \#}}}
% 'st' 'nd' 'rd' 'th' superscripts for numbers
\def\first{{\raise.5ex\hbox{\small st}}}
\def\second{{\raise.5ex\hbox{\small nd}}}
\def\third{{\raise.5ex\hbox{\small rd}}}
\def\fourth{{\raise.5ex\hbox{\small th}}}



%%
%% starting the actual document
%%

\begin{document}

%the name in big fonts at the top of resume
%this is left aligned
\name{Kevin Berry}

%this is right aligned
\address{kevin.berry@villanova.edu}
\address{http://kjb.homeunix.com}

\begin{resume}

%%
%% This section of code is inelegant, but I'm too lazy to fix it
%%

\section{\textsc{Objective}}
A full-time software engineering position in a leading technology company.
\section{\textsc{Education}}

\textbf{Villanova University} \hfill May 2011 \\
BS Computer Science
\newline

%%
%% the meat of the resume starts now
%%

\begin{formatb}
  \employer{l}\title{r}\\
  \location{l}\dates{r}\\
  \body\\
\end{formatb}

\section{\textsc{Professional}}

\employer{\textbf{Google}}
\title{}
\location{Summer of Code - The Tor Project}
\dates{Summer 2010}
\begin{position}
Developed a new database driven back-end for the Tor Metrics Portal to track
statistics, publish data, and create visualizations for the entire Tor network.
I also helped to create a more dynamic and interactive website. \\
{\itshape http://metrics.torproject.org} \\
{\itshape http://github.com/kjbbb}
\end{position}

\employer{\textbf{ESI Medical}}
\title{Software Engineer}
\location{Belmar, NJ}
\dates{Summer 2009}
\begin{position}
Created complete registration and scheduling portal web app for administrators
and employees for flu clinics. It is currently used by hundreds of employees to
upload and encrypt documents, track payroll, register for times and locations.
\\ {\itshape http://esimedical.com/vcr}
\end{position}

\employer{\textbf{Alliant Managed Services}}
\title{Software Engineer}
\location{Middletown, NJ}
\dates{Fall 2010}
\begin{position}
Created internal web application to keep track of managed systems for their
clients.
\end{position}

\employer{\textbf{Various Establishments}}
\title{Bus Boy}
\location{Jersey Shore Area}
\dates{2004-2008}
\begin{position}
Bussed tables at {\itshape Hook, Line \& Sinker} and {\itshape Sea Bright Beach
Club}. Kept the places clean and the cutomers happy.
\end{position}

%%
%% We use the same formatting for projects as for work experience
%% Shown below is the formatting used previously
%%
%%  \begin{formatb}
%%    \employer{l}\title{r}\\
%%    \location{l}\dates{r}\\
%%    \body\\
%%  \end{formatb}
%%
%% 
%%  Note that \location is now being used for non-location information
%%


\begin{formatb}
  \employer{l}\dates{r}\\
  \body\\
\end{formatb}

\section{\textsc{Other Projects}}

\employer{\textbf{Distributed Computing via the Browser}}
\dates{Fall 2009}
\begin{position}
Created a demo and published a paper for a research class which involves
collaboratively reverse hashing a hidden list of words by using the processing
power of the people currently browsing the website. \\
{\itshape http://kevinjberry.com/research}
\end{position}

\employer{\textbf{Sprite Cutter}}
\dates{Spring 2009}
\begin{position}
Developed a CGI application for a software engineering class in C++ to accept an
image file, find the discrete images and sections within it (usually a sprite
sheet), and publish the images and vectors in a useful format. Based on the
OpenCV image processing library. \\
{\itshape http://github.com/kjbbb/spritecutter}
\end{position}


%%
%% This section could also use more formatting, but looks ok, as is
%%

\section{\textsc{Skills}}

\emph{Programming Languages}: Java, C, C++, PHP, SQL, PL/SQL, Python, Ruby, R,
Scheme, Bash, JavaScript

\emph{Libraries and Tools}: Vim, Git, Subversion, ggplot2, Rserve, Apache,
Apache Tomcat, J2EE web stuff, LaTeX, GCC, CodeIgniter, PostgreSQL, MySQL,
Oracle, Unix Utilities, SSH, cURL

\emph{Operating Systems}: Linux (various flavors), Windows, some Darwin

%%
%% Note that we're redefining the formatting
%% We only have one row of information now, instead of two
%%

\section{\textsc{Activities}}

\emph{Elected Captain Villanova Alpine Ski Team} \\
\emph{Elected Captain Rumson Fair-Haven Cross Country Team} \\
\emph{Elected Captain Rumson Fair-Haven Indoor Track Team} \\

\begin{formatb}
  \employer{l}\dates{r}\\
  \body\\
\end{formatb}


%%
%% Nothing special here, just a normal table
%%

\section{\textsc{Course Work}}
  \begin{tabular}{lllll}

  Algorithms             & \ \ & Data Structures     & \ \ & Theory of Computation \\
  Software Engineering   & \ \ & Research Topics     & \ \ & Programming Languages \\
  Statistics             & \ \ & Calculus            & \ \ & Discrete Structures \\
  Databases              & \ \ & Operating Systems   & \ \ & 

  \end{tabular}

\end{resume}
\end{document}
