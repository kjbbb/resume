%%%%%%%%%%%%%%%%%%%%%%%%%%%%%%%%%%%%%%%%%%%%%%%%%%%%%%%%%%%%%%%%%%%
%% 
%% Kevin Berry's resume
%%
%%%%%%%%%%%%%%%%%%%%%%%%%%%%%%%%%%%%%%%%%%%%%%%%%%%%%%%%%%%%%%%%%%%

%%
%% The following code sets up the document formatting
%%

%this assumes that resume.cls is in some path

\documentclass[line,margin]{resume}
\usepackage{graphicx}

\addtolength{\oddsidemargin}{-0.45in}
\addtolength{\voffset}{-0.30in}
\addtolength{\textwidth}{1.00in} \addtolength{\textheight}{1.50in}

\renewcommand{\namefont}{\LARGE\emph}

%%
%% The following code defines some macros for terms which have raised font
%% (ie 4\fourth would result 4th with the 'th' raised (superscripted)
%%

\def\Cplusplus{{\rm C\raise.5ex\hbox{\small ++}}}
\def\CSharp{{\rm C\raise.5ex\hbox{\small \#}}}
% 'st' 'nd' 'rd' 'th' superscripts for numbers
\def\first{{\raise.5ex\hbox{\small st}}}
\def\second{{\raise.5ex\hbox{\small nd}}}
\def\third{{\raise.5ex\hbox{\small rd}}}
\def\fourth{{\raise.5ex\hbox{\small th}}}

%%
%% starting the actual document
%%

\begin{document}

\begin{format}
  \employer{l}\title{r}\\
  \location{l}\dates{r}\\
  \body\\
\end{format}
%the name in big fonts at the top of resume
%this is left aligned
\name{Kevin Berry}

%this is right aligned
%\address{kberry01@villanova.edu}
%\address{http://kjb.homeunix.com}

\begin{resume}
\begin{flushright}
2 South Rohallion Drive, Rumson, NJ 07760 \\
kevin.berry@villanova.edu \\
http://kjb.homeunix.com \\
http://github.com/kjbbb\\
732-492-1206 \\
\end{flushright}
%%
%% This section of code is inelegant, but I'm too lazy to fix it
%%

\section{\textsc{Objective}}
A full-time software engineering position at a leading technology company.
Interested in a wide range of software engineering and computer science topics,
from systems and security to distributed systems to web development.
\section{\textsc{Education}}

\textbf{Villanova University} \hfill May 2011 \\
BS Computer Science
%\newline

%%
%% the meat of the resume starts now
%%

\section{\textsc{Professional}}

\employer{\textbf{Google}}
\title{\emph{Software Engineer}}
\location{\emph{Summer of Code - The Tor Project}}
\dates{\emph{Summer 2010}}
\begin{position}
Successfully participated in Google's open-source student program to develop a
database driven back-end for the Tor Metrics Portal to track statistics, publish
data, and create visualizations for the entire Tor network. I also helped to
create a more dynamic and interactive website. The technologies included
PostgreSQL, GNU R, ggplot2, Apache Tomcat, and Java J2EE.\\
\emph{http://metrics.torproject.org} \\
\emph{http://github.com/kjbbb/metrics-web, http://github.com/kjbbb/metrics-db}
\end{position}

\employer{\textbf{ESI Medical}}
\title{\textit{Software Engineer}}
\location{\textit{Belmar, NJ}}
\dates{\textit{Summer 2009}}
\begin{position}
Created complete registration and scheduling portal web app for administrators
and employees for flu clinics. It is currently used by hundreds of employees to
upload and encrypt documents, track payroll, and register for times and
locations. The technologies included PHP, Apache, MySQL, imagemagick, and
Javascript (jQuery).
\\ {\itshape http://esimedical.com/vcr}
\end{position}

\employer{\textbf{Alliant Managed Services}}
\title{\textit{Software Engineer}}
\location{\textit{Morristown, NJ}}
\dates{\textit{Fall 2010}}
\begin{position}
Created internal web application to keep track of managed systems for their
clients. Technologies included PHP, Apache, SOAP, CodeIgniter, and JavaScript
(jQuery).
\end{position}

\section{\textsc{Other Projects}}

\begin{format}
  \employer{l}\dates{r}\\
  \body\\
\end{format}

\employer{\textbf{\\Distributed Computing via the Browser}}
\dates{\textit{\\Fall 2009}}
\begin{position}
Created a demo and published a paper for a research class which involved
collaboratively reverse hashing a hidden list of words by using the processing
power of the people currently browsing the website.
\end{position}

\employer{\textbf{Sprite Cutter}}
\dates{\textit{Spring 2009}}
\begin{position}
Developed an application for a software engineering class in C++ to accept an
image file, find the discrete images and sections within it (usually a sprite
sheet), and publish the images and vectors in a useful format. Based on the
OpenCV image processing library. \\
{\itshape http://github.com/kjbbb/spritecutter-web}
\end{position}

%%
%% This section could also use more formatting, but looks ok, as is
%%

\section{\textsc{Skills}}

\emph{Programming Languages}: Java, C, C++, PHP, SQL, PL/SQL, Python, Ruby, R,
Scheme, Bash, JavaScript, AWK, asm (debugging i386, x86-64)

\emph{Libraries and Tools}: vim, git, subversion, ggplot2, Rserve, Apache,
Apache Tomcat, J2EE web, LaTeX, gcc, gdb, gas, CodeIgniter, PostgreSQL,
MySQL, Oracle, unix utilities, debugging tools, ssh, cURL

\emph{Operating Systems}: Unix - Proficiency with Linux (Debian, Arch), OSX,
familiararity with BSDs, Solaris, Windows

%%
%% Note that we're redefining the formatting
%% We only have one row of information now, instead of two
%%

\section{\textsc{Activities}}

\emph{Elected Captain, Treasurer, and President of Villanova Alpine Ski Team} \\
\emph{Elected Captain Rumson Fair-Haven Cross Country Team} \\
\emph{Elected Captain Rumson Fair-Haven Indoor Track Team} \\
\newpage

\end{resume}
\end{document}
